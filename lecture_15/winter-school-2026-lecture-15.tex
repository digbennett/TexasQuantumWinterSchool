\documentclass[aps,jcp,longbibliography,floatfix,reprint,onecolumn,12pt]{revtex4-2}
\usepackage{amsfonts, amsmath, amsthm, amssymb, physics}
\usepackage{graphicx}
\usepackage[margin=0.75in]{geometry}
\usepackage{xcolor}
\usepackage[skins,theorems]{tcolorbox}
\usepackage{graphicx}
\usepackage{mathtools}
\usepackage{wrapfig}
\usepackage{dsfont}
\usepackage{hyperref}
\usepackage{enumitem}
\hypersetup{colorlinks, 
	linkcolor={blue!75!black!80!yellow},
	citecolor={blue!75!black!80!yellow}, 
	urlcolor={blue!75!black!80!yellow}
	}
	
\usepackage{fancyheadings}
\pagestyle{fancy}
\rhead{Lecture 15: Texas Quantum Winter School 2026 \hspace{49mm} Kade Head-Marsden}
\definecolor{beaublue}{rgb}{0.74, 0.83, 0.9}
\definecolor{pastelblue}{rgb}{0.68, 0.78, 0.81}
\def\bibsection{}

\newtcolorbox{exerciseEnv}{
center,
width=\linewidth,
colframe=pastelblue,
colback=white
}

%\title{Lectures 3 and 4: Loss of information with Kraus maps\\ \large{\emph{Texas Quantum Winter School on Open Quantum Systems and Quantum Information in the Chemical Sciences} }}
%\author{Kade Head-Marsden \\ University of Minnesota - Twin Cities}
%\date{January 5-8, 2026}

\begin{document}
%\maketitle

\section*{Quantum computing}

\begin{tcolorbox}[title = Learning Objective]
\begin{center}
   Understand how to read quantum circuits
\end{center}
\end{tcolorbox}

\subsection*{States}
\begin{itemize}[noitemsep, topsep = 0pt]
    \item A single-qubit state can be represented on a Bloch sphere and written as,
    \begin{equation}
        \ket{\psi} = \cos(\frac{\theta}{2})\ket{0} + e^{i\phi}\sin(\frac{\theta}{2})\ket{1}
    \end{equation}
    \item This picture works for a static, single qubit picture but if we want to look at gates acting on one or more qubits, often we turn to circuit diagrams
\end{itemize}

\subsection*{Gates supported on a single qubit}
\begin{itemize}[noitemsep, topsep = 0pt]
    \item \textbf{Unitary gates} are how we operate on qubits in gate-based quantum computing. A unitary gate in a circuit is equivalent to a unitary matrix or operator in traditional mathematics or quantum mechanics
    \item An example of a few \textbf{single-qubit gates} acting on a qubit state $\ket{q_0} = \ket{0}$:
    \begin{center}
        \includegraphics[width=\linewidth]{lecture-15-single-qubit-gates.png}
    \end{center}
    \item Note here that if you have multiple qubits, you will have multiple qubit rails and the state will be given by a tensor product of qubits which we can write as 
    \begin{equation}
        \ket{\psi} = \ket{q_0}\otimes\ket{q_1}
    \end{equation}
    for two qubits or as,
    \begin{equation}
        \ket{\psi} = \bigotimes_i \ket{q_i}
    \end{equation}
    for more qubits
    \item As an example circuit with two qubits, we have:
    \begin{center}
        \includegraphics[width = 0.3\textwidth]{lecture-15-2-qubit-system-single-qubit-gates.png}
    \end{center}
    Mathematically we can write this as a produce of each operator,
    \begin{align}
        \ket{\psi} &= (U_1\otimes \mathds{1})(U_0\otimes \mathds{1}) (\ket{q_0}\otimes\ket{q_1})\\
        \notag
        &= U_1U_0\ket{q_0}\otimes \mathds{1}\mathds{1}\ket{q_1}
    \end{align}
    \item In the above we use the \textbf{mixed product property} of the Kronecker product,
    \begin{equation}
        (A\otimes B)(C\otimes D) = AC\otimes BD
    \end{equation}
    \item \emph{\textbf{Exercise 1}}
    \item If a state can be written as a single Kronecker product, then we refer to the state as \textbf{separable}. Separable gates are also referred to as \textbf{gates supported on one qubit} 
    \item The \textbf{Pauli gates} are a common gate set,
    \begin{equation}
        X = \begin{pmatrix}
            0 & 1\\ 1 & 0
        \end{pmatrix} \qquad Y = \begin{pmatrix}
            0 & -i\\ i & 0
        \end{pmatrix} \qquad Z = \begin{pmatrix}
            1 & 0\\ 0 & -1 
        \end{pmatrix}
    \end{equation}
    \item Single-qubit \textbf{rotation gates} are also constructed from exponentiating the Pauli operators,
    \begin{align}
        R_X(\theta) &= e^{-\frac{iX\theta}{2}} = \begin{pmatrix}
            \cos(\frac{\theta}{2}) & -i\sin(\frac{\theta}{2})\\
            -\sin(\frac{\theta}{2}) & \cos(\frac{\theta}{2})\\
        \end{pmatrix}\\
        \notag
        R_Y(\theta) & = e^{-\frac{iY\theta}{2}} = \begin{pmatrix}
            \cos(\frac{\theta}{2}) & -\sin(\frac{\theta}{2})\\
            \sin(\frac{\theta}{2}) & \cos(\frac{\theta}{2})\\
        \end{pmatrix}\\
        \notag
        R_Z(\theta) &= e^{-\frac{iY\theta}{2}} = \begin{pmatrix}
            e^{-\frac{i\theta}{2}} & 0\\
            0 &  e^{\frac{i\theta}{2}}\\
        \end{pmatrix}
    \end{align}
    \item The \textbf{Hadamard} gate is another very common gate,
    \begin{equation}
        H = \frac{1}{\sqrt{2}}\begin{pmatrix}
            1 & 1\\ 1 & -1
        \end{pmatrix}
    \end{equation}
\end{itemize}

\subsection*{Gates supported on multiple qubits}
\begin{itemize}[noitemsep, topsep = 0pt]
    \item Also referred to as 2-qubit or $k$-qubit gates
    \item The most common 2-qubit gate is a CNOT (controlled-not) gate,
    \begin{equation}
        CNOT = \begin{pmatrix}
            1 & 0 & 0 & 0\\
            0 & 1 & 0 & 0\\
            0 & 0 & 0 & 1\\
            0 & 0 & 1 & 0
        \end{pmatrix}
    \end{equation}
    if we order our 2-qubit basis $\{ 00,01,10,11\}$ and is represented in a circuit as:
    \begin{center}
    \includegraphics[width = 0.25\textwidth]{lecture-15-cnot.png}
    \end{center}
    \item If we act this CNOT gate on the state where both qubits are initially in their ground state then,
    \begin{align}
        \ket{\psi} &= \begin{pmatrix}
            1 & 0 & 0 & 0\\
            0 & 1 & 0 & 0\\
            0 & 0 & 0 & 1\\
            0 & 0 & 1 & 0
        \end{pmatrix} \ket{0}\otimes\ket{0}\\
        \notag
        &= \begin{pmatrix}
            1 & 0 & 0 & 0\\
            0 & 1 & 0 & 0\\
            0 & 0 & 0 & 1\\
            0 & 0 & 1 & 0
        \end{pmatrix} \begin{pmatrix}
            1 \\ 0\\ 0\\ 0\\
        \end{pmatrix}\\
        \notag
        &= \begin{pmatrix}
            1 \\ 0\\ 0\\ 0\\
        \end{pmatrix}
    \end{align}
    Physically, we are saying that the second qubit state depends on the first. If the first qubit is in state $\ket{0}$ then nothing will happen to the second qubit, which is exactly what we see here.
    \item Note in the above that one needs to be very careful about the basis for multiple qubits -- different software will order basis states differently.
    \item \emph{\textbf{Exercise 2}}
    \item A more general 2-qubit control gate is given by,
    \begin{equation}
        CU = \begin{pmatrix}
            1 & 0 & 0 & 0\\
            0 & 1 & 0 & 0\\
            0 & 0 & u_{00} & u_{01}\\
            0 & 0 & u_{10} & u_{11}
        \end{pmatrix}
    \end{equation}
    and is represented in the following circuit:
    \begin{center}
        \includegraphics[width = 0.25\textwidth]{lecture-15-cu.png}
    \end{center}
    If the \textbf{control qubit} is in state $\ket{0}$ then nothing happens to the second qubit (the \textbf{target qubit}), if the control qubit is in state $\ket{1}$ then the operator $U$ acts on the target qubit
    \item Sometimes control qubits are only used to alter the target qubits that are more relevant for our algorithm. Qubits that are used to facilitate a computation but do not contain necessary information are often referred to as \textbf{ancilla} or \textbf{ancillary qubits}
\end{itemize}

\subsection*{Circuit scaling and things to keep in mind}
\begin{itemize}[noitemsep, topsep = 0pt]
    \item We are currently in the \textbf{noisy-intermediate scale quantum (NISQ) era} which is defined by small qubit numbers where each qubit is individually fairly prone to errors
    \item Often NISQ algorithms report complexity in terms of CNOT complexity because these errors are the limiting factor
    \item The circuit order is the opposite from the ordering of mathematical multiplication
    \item \textbf{Circuit depth:} The number of gates that occur (length of circuit)
    \item \textbf{Circuit volume:} Combination of number of qubits and number of gates
    \item Connectivity of the device is relevant for circuit depth
    \item \textbf{Native gate set:} which gates are physically realizable on a given hardware platform; relevant for circuit depth
    \item \textbf{Transpilation:} Breaking a unitary down into native gates
    \item An example of where connectivity matters where the left is a circuit and the right is the hypothetical connection of three qubits:
    \begin{center}
        \includegraphics[width = 0.67\textwidth]{lecture-15-cnot-cnot.png}
    \end{center}
    The way this is written, when the circuit is implemented, a SWAP gate will be necessary:
    \begin{center}
        \includegraphics[width = 0.67\textwidth]{lecture-15-swap-gate.png}
    \end{center}
    where the SWAP gate exchanges the states,
    \begin{equation}
        SWAP\big(\ket{q_1}\otimes\ket{q_2}\big) = \ket{q_2}\otimes\ket{q_1}
    \end{equation}
    \item \emph{\textbf{Exercise 3}}  
\end{itemize}

\subsection*{Measurement}

\begin{itemize}[noitemsep, topsep = 0pt]
    \item The final step of a quantum circuit is referred to as \textbf{measurement} where the state needs to be collapsed to obtain information, which is depicted in the circuit symbol below:
    \begin{center}
        \includegraphics[width = 0.3\textwidth]{lecture-15-measurement.png}
    \end{center}
    \item The process of measuring all the density matrix elements is referred to as \textbf{tomography}
    \item There are many more modern approaches to extract information that bypass full tomography 
    \item Even with perfect qubits, this is still a statistical process and therefore many samples need to be taken to reproduce a quantum state  (think flipping a coin, rolling dice)
\end{itemize}


\end{document}