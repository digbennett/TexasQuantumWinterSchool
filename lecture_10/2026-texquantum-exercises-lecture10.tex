\documentclass{article}

\usepackage{microtype}

\usepackage{mathtools}

\usepackage{enumitem}
\usepackage{amsthm}
\usepackage{amssymb}
\usepackage{derivative}

\newtheorem{exercise}{Exercise}
\newenvironment{solution}{\begin{proof}[Solution]}{\end{proof}}

\newcommand{\Comp}{\mathbb{C}}
\newcommand{\tens}{\otimes}

\DeclareMathOperator{\Tr}{Tr}                               %              Trace
\DeclarePairedDelimiter\abs{\lvert}{\rvert}                 %                |x|

%%%%%%%%%%%%%%%%%%%%%%%%%%%%%%%%%%%%%%%%%%%%%%%%%%%%%%%%%%%%%%%%%%%%%%%%%%%%%%%%
%%%%                            Dirac notation                              %%%%
%%%%%%%%%%%%%%%%%%%%%%%%%%%%%%%%%%%%%%%%%%%%%%%%%%%%%%%%%%%%%%%%%%%%%%%%%%%%%%%%
\DeclarePairedDelimiter\bra{\langle}{\rvert}               %                 <a|
\DeclarePairedDelimiter\ket{\lvert}{\rangle}               %                 |a>
\DeclarePairedDelimiterX\ip[1]{\langle}{\rangle}
  {#1 \delimsize| #1}                                      %               <a|a>
\DeclarePairedDelimiterX\op[1]{\lvert}{\rvert} 
  {#1 \delimsize\rangle\!\delimsize\langle #1}             %              |a><a|
\DeclarePairedDelimiterX\braket[2]{\langle}{\rangle}
  {#1 \delimsize| #2}                                      %               <a|b>
\DeclarePairedDelimiterX\ketbra[2]{\lvert}{\rvert} 
  {#1 \delimsize\rangle\!\delimsize\langle #2}             %              |a><b|
\DeclarePairedDelimiter\evo{\langle}{\rangle}              %                 <O>
\DeclarePairedDelimiterX\ev[2]{\langle}{\rangle}
  {#2 \delimsize| #1 \delimsize| #2}                       %             <a|O|a>
\DeclarePairedDelimiterX\mel[3]{\langle}{\rangle}
  {#1 \delimsize| #2 \delimsize| #3}                       %             <a|O|b>

\DeclarePairedDelimiter\normalord{:}{:}

\title{Exercises: Stochastic methods for open quantum systems I}
\author{Texas Winter Quantum School}
\date{2026-01-07}

\begin{document}

\maketitle

\begin{exercise}[Wide open system limit]
Consider a chain of four \emph{one-level} quantum systems in the
system-bath coupling limit: each site $k$ is coupled to a bath with
one Lindblad operator $L_k = \op{k}$. An arbitrary initial state is
\begin{equation}
    \ket{\psi} = \sum_{k=1}^4 c_k \ket{k}; \quad
    c_k \in \Comp, \ \textstyle\sum_k \abs{c_k}^2 = 1.
\end{equation}

Define the \emph{delocalization entropy}
\begin{equation}
    S(\psi) = -\sum_k \abs{c_k}^2 \ln{\left(\abs{c_k}^2\right)}.
\end{equation}

\begin{enumerate}[label=(\alph*)]
    \item Show that $S = 0$ for $\ket{\psi} = \ket{k}$;
    \item Compute $S(\psi)$ for the fully delocalized state
    \begin{equation}
        \ket{\psi} = 
        \frac{1}{\sqrt{4}} \left(\ket{1} + \ket{2} + \ket{3} + \ket{4} \right).
    \end{equation}
    Is it higher or lower than for the localized state?
    \item In a wide open system, the dispersion entropy theorem states that
          \begin{equation}
            \odv*{\mathcal{M}_z[S(\ket{\psi_z(t)}]}{t} = 
            -\sum_k \frac{1 - \abs{c_k}^2}{\abs{c_k}^2} \ev{L_k}{\psi_z} \leq 0.
          \end{equation}
          What does this mean for the average delocalization of the wavefunction
          as a function of time? What should happen if the system starts in a
          totally delocalized state? A localized state?
    \item Returning to our original formulation of the problem, observe that
          $\ket{\Psi}$ is the total wave function for the system and bath. Explain
          what your result for (c) means in terms of this total wave
          function. What does the dispersion entropy theorem say in terms of the
          total wave function?
\end{enumerate}
\end{exercise}

\begin{solution}
\hphantom{Beginning of solution} \\
\begin{enumerate}[label=(\alph*)]
    \item For a state fully localized on site $k$, $c_k = 1$ and $c_i = 0$
          ($i \neq k$). So
          \begin{equation}
          \begin{split}
              S(\ket{k}) &=
              - \abs{c_k}^2 \ln{(\abs{c_k}^2)} - 
                \sum_{i \neq k} \abs{c_i}^2 \ln{(\abs{c_i}^2)} \\ &=
              -(1 \ln{1}) - \sum_{i \neq k} (0 \ln{0}) \\ 
              &= 0.
          \end{split}
          \end{equation}
    \item In the totally delocalized state, $c_k = 1/\sqrt{4} = 1/2$ for all $k$,
          and
          \begin{equation}
              S(\ket{\psi}) = -4 \left( \frac14 \ln{\frac14} \right) =
              -\ln{\frac14} = \ln{4}.
          \end{equation}
          (More generally, for $d$ sites, $c_k = 1/\sqrt{d}$, and
          $S(\ket{\psi}) = \ln{d}$.) This is strictly positive, and in particular
          greater than the delocalization entropy of the localized state 
          $\ket{k}$.
    \item The entropy of any $\ket{\psi_z}$ decreases over time;
          thus, the delocalized state will evolve towards a point of
          minimal entropy (a fully localized state $\ket{k}$, which has the
          minimum $S = 0$). In $\ket{k}$, the entropy cannot decrease,
          so the system must stay there;
          since there are no couplings between sites,
          either in the system Hamiltonian or in the bath operators,
          it will stay in $\ket{k}$ for all time.
    \item Despite the fact that the system is completely uncoupled,
          the stochastic evolution of the bath drives fluctuations in the
          amplitudes $c_k$. Eventually, one will become larger than the others,
          and the system will localize on that site, effectively transferring
          its delocalization entropy to the bath.
\end{enumerate}
\end{solution}

\begin{exercise}[Limitations of non-Markovian quantum state diffusion]
Looking back over this method, what are the major limitations of the method
presented? We spoke at the very beginning about the long timescale of method
development, so there have been a lot of developments since this method was first
developed - what are the key developments you would want to go look for in the
literature if you were interested in this method?
\end{exercise}

\begin{solution}
Limitations include
\begin{enumerate}
    \item The functional derivative in the non-Markovian kernel is not at all
          amenable to easy calculation.
    \item We used Bargmann states in our derivation, so this equation is
          propagating an unnormalized wavefunction. This will lead to difficulty
          interpreting amplitudes as connected to probabilities.
    \item We evolve only pure states, not general density matrices.
\end{enumerate}
In extensions of the method, I would look for solutions to some or all of
these problems.
\end{solution}

\end{document}
