\documentclass[aps,jcp,longbibliography,floatfix,reprint,onecolumn,12pt]{revtex4-2}
\usepackage{amsfonts, amsmath, amsthm, amssymb, physics}
\usepackage{graphicx}
\usepackage[margin=0.75in]{geometry}
\usepackage{xcolor}
\usepackage[skins,theorems]{tcolorbox}
\usepackage{graphicx}
\usepackage{mathtools}
\usepackage{wrapfig}
\usepackage{dsfont}
\usepackage{hyperref}
\usepackage{enumitem}
\hypersetup{colorlinks, 
	linkcolor={blue!75!black!80!yellow},
	citecolor={blue!75!black!80!yellow}, 
	urlcolor={blue!75!black!80!yellow}
	}
	
\usepackage{fancyheadings}
\pagestyle{fancy}
\rhead{Lectures 3 \& 4: Texas Quantum Winter School 2026 \hspace{40mm} Kade Head-Marsden}
\definecolor{beaublue}{rgb}{0.74, 0.83, 0.9}
\definecolor{pastelblue}{rgb}{0.68, 0.78, 0.81}


\newtcolorbox{exerciseEnv}{
center,
width=\linewidth,
colframe=pastelblue,
colback=white
}

%\title{Lectures 3 and 4: Loss of information with Kraus maps\\ \large{\emph{Texas Quantum Winter School on Open Quantum Systems and Quantum Information in the Chemical Sciences} }}
%\author{Kade Head-Marsden \\ University of Minnesota - Twin Cities}
%\date{January 5-8, 2026}

\begin{document}
%\maketitle

\section*{Exercise 1: Write out the matrix multiplication represented by the circuit diagram:}
\begin{center}
        \includegraphics[width = 0.3\textwidth]{lecture-15-single-qubit-gates.png}
\end{center}

\noindent \textbf{Solution:}
    \begin{align*}
        \ket{\psi} &= (U_1\otimes \mathds{1})(\mathds{1}\otimes U_2)(U_0\otimes \mathds{1}) (\ket{q_0}\otimes\ket{q_1})\\
        \notag
        &= U_1U_0\ket{q_0}\otimes U_2\ket{q_1}
    \end{align*}

\section*{Exercise 2: Write out the matrix multiplication represented by the circuit diagram:}
\begin{center}
\includegraphics[width = 0.25\textwidth]{lecture-15-x-cnot.png}
\end{center}

\noindent \textbf{Solution:}
\begin{align*}
    \ket{\psi} &= \begin{pmatrix}
            1 & 0 & 0 & 0\\
            0 & 1 & 0 & 0\\
            0 & 0 & 0 & 1\\
            0 & 0 & 1 & 0
        \end{pmatrix} \bigg( X\ket{0}\otimes\ket{0} \bigg)\\
        &= \begin{pmatrix}
            1 & 0 & 0 & 0\\
            0 & 1 & 0 & 0\\
            0 & 0 & 0 & 1\\
            0 & 0 & 1 & 0
        \end{pmatrix} \bigg( \ket{1}\otimes\ket{0} \bigg)\\
        &= \begin{pmatrix}
            1 & 0 & 0 & 0\\
            0 & 1 & 0 & 0\\
            0 & 0 & 0 & 1\\
            0 & 0 & 1 & 0
        \end{pmatrix} \begin{pmatrix}
            0 \\ 0\\ 1\\ 0\\
        \end{pmatrix}\\
        \notag
        &= \begin{pmatrix}
            0 \\ 0\\ 0\\ 1\\
        \end{pmatrix}
\end{align*}

\section*{Exercise 3: Consider the following circuit and qubit layout:}
\begin{center}
        \includegraphics[width = 0.67\textwidth]{lecture-15-cnot-cnot.png}
\end{center}
A natural mapping would be to say that the qubit furthest to the left is $\ket{q_0}$, the middle one is $\ket{q_1}$, and the one on the right is $\ket{q_2}$. Why would this be a bad mapping?

\noindent \textbf{Solution:} In the circuit, $\ket{q_0}$ needs to interact directly with both $\ket{q_1}$ and $\ket{q_2}$. It would be easier to have this qubit be in the center so it has direct connectivity for the required gates.

\section*{Exercise 4: Why is the Kraus operator harder to map into quantum gates than the Hamiltonian?}

\noindent \textbf{Solution:} $M_k$ is not necessarily unitary! If there is only one $M_k$ then $M_k^{\dagger}M_k = \mathds{1}$; however that is the exception and not the norm. Generally, $\sum_k M_k M_k^{\dagger} = \mathds{1}$ but each individual operator is not unitary.

\section*{Exercise 5: Check that $S$ is symmetric and $A$ is antisymmetric}

\noindent \textbf{Solution:} 
\begin{align*}
    S^\dagger = \bigg( \frac{M + M^\dagger}{2} \bigg)^{\dagger} = \frac{M^{\dagger} + M}{2} = S
\end{align*}

\begin{align*}
    A^\dagger = \bigg( \frac{M - M^\dagger}{2} \bigg)^{\dagger} = \frac{M^{\dagger} - M}{2} = -A
\end{align*}

\section*{Exercise 6: Verify the Taylor expansion versions of $S$ and $A$}

\noindent \textbf{Solution:} 

For $S$:
\begin{align*}
    e^{-i\epsilon S} &= 1- i\epsilon S + \mathcal{O}(\epsilon^2)\\
    e^{i\epsilon S} &= 1+ i\epsilon S + \mathcal{O}(\epsilon^2)
\end{align*}
then take the difference and reshuffle,
\begin{align*}
    e^{-i\epsilon S} - e^{i\epsilon S} &= 1- i\epsilon S + \mathcal{O}(\epsilon^2) - (1+ i\epsilon S + \mathcal{O}(\epsilon^2))\\
     e^{-i\epsilon S} - e^{i\epsilon S} &= - 2i\epsilon S  \\
     S &= \frac{1}{-2i\epsilon} \bigg(e^{-i\epsilon S} - e^{i\epsilon S}\bigg)\\
     S &= \frac{i}{2\epsilon} \bigg(e^{-i\epsilon S} - e^{i\epsilon S}\bigg)\\
\end{align*}

For $A$:
\begin{align*}
    e^{\epsilon A} &= 1+ \epsilon A + \mathcal{O}(\epsilon^2)\\
    e^{-\epsilon A} &= 1- \epsilon A + \mathcal{O}(\epsilon^2)
\end{align*}
then take the difference and reshuffle,
\begin{align*}
    e^{\epsilon A} - e^{-\epsilon A} &= 1+ \epsilon A + \mathcal{O}(\epsilon^2) - (1- \epsilon A + \mathcal{O}(\epsilon^2))\\
     e^{\epsilon A} - e^{\epsilon A} &= 2\epsilon A  \\
     A &= \frac{1}{2\epsilon} \bigg(e^{\epsilon A} - e^{-\epsilon A}\bigg)\\
\end{align*}

\section*{Exercise 7: Verify these four opreators are unitary.}

\noindent \textbf{Solution:} 

\begin{align*}
    (ie^{-i\epsilon S})^\dagger ie^{-i\epsilon S} &=  -ie^{i\epsilon S^\dagger} ie^{-i\epsilon S} =   e^{i\epsilon S}e^{-i\epsilon S} = \mathds{1} \\
    (-ie^{i\epsilon S})^\dagger (-i)e^{i\epsilon S} &=  ie^{-i\epsilon S^\dagger} (-i)e^{i\epsilon S} =   e^{-i\epsilon S}e^{i\epsilon S} = \mathds{1} \\
    (e^{\epsilon A})^\dagger e^{\epsilon A} &=  (e^{\epsilon A^{\dagger}}) e^{\epsilon A} =   e^{-\epsilon A}e^{\epsilon A} = \mathds{1} \\
    (e^{-\epsilon A})^\dagger e^{-\epsilon A} &=  (e^{-\epsilon A^{\dagger}}) e^{-\epsilon A} =   e^{\epsilon A}e^{-\epsilon A} = \mathds{1} \\
\end{align*}

\section*{Exercise 8: Linear combination of unitaries} 

Consider a simpler problem,
\begin{align}
    \begin{pmatrix}
        U_0 & 0 \\ 0 & U_1 
    \end{pmatrix} \begin{pmatrix}
        \ket{\psi}\\ \ket{\psi}
    \end{pmatrix} = \begin{pmatrix}
        U_0\ket{\psi}\\ U_1\ket{\psi}
    \end{pmatrix}
\end{align}
The probability of measuring the ancilla in the state $\ket{0}$ is $P(0) = \bra{\psi} U_0^{\dagger} U_0 \ket{\psi}$ and in the state $\ket{1}$, $P(1) = \bra{\psi} U_1^{\dagger} U_1 \ket{\psi}$. These are not the same thing as $\bra{\psi} M^{\dagger}M \ket{\psi}$ because this requires a sum of the unitaries. So what can we do differently? 

\noindent \textbf{Solution:} Rotation! 

\begin{align}
    \begin{pmatrix}
        \mathds{1} & \mathds{1} \\
        \mathds{1} & -\mathds{1} 
    \end{pmatrix}\begin{pmatrix}
        U_0 & 0 \\ 0 & U_1
    \end{pmatrix} \begin{pmatrix}
        \ket{\psi}\\ \ket{\psi}
    \end{pmatrix} &= \begin{pmatrix}
        \mathds{1} & \mathds{1} \\
        \mathds{1} & -\mathds{1} 
    \end{pmatrix}\begin{pmatrix}
        U_0\ket{\psi}\\ U_1\ket{\psi}
    \end{pmatrix}\\
    &= \begin{pmatrix}
        (U_0+U_1)\ket{\psi}\\ (U_0-U_1)\ket{\psi}
    \end{pmatrix}
\end{align}

\end{document}